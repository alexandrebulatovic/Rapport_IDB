\section{Les fonctionnalités d'un SGBD}
L'application propose des fonctionnalités vues durant le cursus à l'IUT.

\subsection{Le \gls{ldd}}
Le langage de définition des données permet de structurer une base de données.
Il ne s'intêresse pas aux données contenues dans les tables
\footnote{\label{interet_ldd}Le LDD prend en compte les données contenues dans les tables et peut agir dessus, mais c'est une conséquence, pas son rôle.}, mais aux tables elles même. Le LDD est séparé en trois instructions:

\subsubsection{CREATE TABLE}
Permet de créer une \gls{table}. Dans un SGBD, les tables sont nommées, chaque nom est unique.
Une table est créée avec au moins un attribut dont il faut préciser le type de données (texte, nombre, date etc.) et la taille.
Des \glspl{constraint} supplémentaires peuvent être ajoutées, comme par exemple les \glspl{primarykey} \footnote{\label{contrainte_clée_primaire}Voir glossaire.}ou encore des \textit{NOT NULL}.

La \gls{query} suivante montre la création d'une table nommée PERSONNES, qui contient les attributs \textit{idpersonne}, \textit{nompersonne}, \textit{taillepersonne} et \textit{datenaissancepersonne}.

  \begin{lstlisting}
    CREATE TABLE PERSONNES
    (
    idpersonne CHAR(5),
    nompersonne VARCHAR(30),
    taillepersonne NUMBER,
    datenaissancepersonne DATE
    );
  \end{lstlisting}


\subsubsection{ALTER TABLE}
Permet de revenir sur ce qui a été fait avec CREATE TABLE.
L'instruction permet d'ajouter, supprimer ou modifier des \glspl{attribut}, des contraintes, des index...

Cette instruction se comporte différemment selon qu'une table soit vide ou remplie de lignes de données (\glspl{tuple}).
Par exemple, ajouter une contrainte NOT NULL sur une colonne possédant déjà des tuples nuls n'est pas possible. Ce problème n'existe pas sur une table vide.

La requête SQL suivante modifie la table \textit{PERSONNES} pour y ajouter une contrainte de clée primaire nommée \textit{pk\_personnes} sur l'attribut \textit{idpersonne}.

\begin{lstlisting}
  ALTER TABLE PERSONNES
  (
  ADD CONSTRAINT pk_personnes PRIMARY KEY (idpersonne)
  );
\end{lstlisting}

\subsubsection{DROP TABLE}
Permet de supprimer une table et les données qu'elle contient.
Dans une base de données relationnelles, les tables sont liées entre elles par des attributs.
La supression d'une table peut entraîner la supression de données dans d'autres tables, en fonction du schéma relationnel de la base.

La requête SQL suivante supprime la table \textit{PERSONNES} de la base de données.
\begin{lstlisting}
  DROP TABLE PERSONNES;
\end{lstlisting}

\subsection{Le \gls{lmd}}
Le langage de manipulation des données permet d'affectuer des actions de \gls{crud} sur ce que contiennent les tables.
En d'autres termes, il agit sur les tuples.

\subsubsection{Create}
Il d'agit de créer un nouveau tuple dans une table.
La requête SQL suivante permet d'ajouter un tuple de clée primaire \textit{00001} dans la table \textit{PERSONNES}.
\begin{lstlisting}
  INSERT INTO PERSONNES
  (idpersonne, nompersonne, taillepersonne,
  datenaissancepersonne)
  VALUES
  ('00001', 'DUPONT', 'Jean', '06/08/1985');
\end{lstlisting}

\subsubsection{Read}
Il s'agit de récupérer, lire, croiser des données que contiennent les tables.
Les requêtes SQL  "read" peuvent être très complexes.
Certains SGBD proposent des \glspl{qbe}
\footnote{Access, LOBase, phpMyAdmin...}pour créer ces requêtes sans manipuler de SQL.
Celle qui est écrite juste après est simple et permet de retrouver le nom de la \textit{PERSONNES} numéro "00001".
\begin{lstlisting}
  SELECT PERSONNES.nompersonne
  FROM PERSONNES
  WHERE PERSONNES.idpersonne = '00001';
\end{lstlisting}

\subsubsection{Update}
Il s'agit de modifier un ou plusieurs tuples qui existent déjà dans la base de données.
La requête SQL suivante remplace le nom de famille de la \textit{PERSONNES} "00001" par "Robert".
\begin{lstlisting}
  UPDATE PERSONNES
  SET nompersonne = 'Robert'
  WHERE idpersonne = '00001';
\end{lstlisting}

\subsubsection{DELETE}
Il s'agit de supprimer un ou plusieurs tuples.
La requête SQL suivante supprime les tuples des \textit{PERSONNES} qui s'appellent "Jean".
\begin{lstlisting}
  DELETE FROM PERSONNES
  WHERE prenompersonne = "Jean";
\end{lstlisting}

\subsection{Le Langage de Controle des Données (LCD)}
Cet aspect du SQL n'est pas demandé pour l'application et n'est pas traité.

\subsection{Transaction ACID}
Cet aspect des bases de données n'est pas demandé pour l'application et n'est pas traité.
En d'autres termes, une base de données ne peut être utilisée que par une seule personne à fois, sinon elle perd sa cohérence.
