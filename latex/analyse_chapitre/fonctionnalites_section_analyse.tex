L'application propose certaines fonctionnalités des SGBD vues durant le cursus à l'IUT.
Un exemple de requête SQL est indiqué pour chaque fonctionnalité.
L'utilisateur \underline{n'a pas} à écrire ces requêtes pour gérer sa base de données.

\subsection{Le Langage de Définition des Données (LDD)}
Le langage de définition des données permet de structurer une base de données.
Il ne s'intéresse pas aux données contenues dans les tables
\footnote{\label{interet_ldd}Le LDD prend en compte les données contenues dans les tables et peut agir dessus, mais c'est une conséquence, pas son rôle.}, mais à la structure des tables. Le LDD est séparé en trois instructions:

\subsubsection{Create table}
Permet de créer une \gls{table}. Dans un SGBD, les tables sont nommées, chaque nom est unique.
Une table est créée avec au moins un attribut dont il faut préciser le type de données (texte, nombre, date etc.) et la taille.
Des \glspl{constraint} supplémentaires peuvent être ajoutées : les \textit{\glspl{primarykey}}, les attributs \textit{not null} et groupes d'attributs \textit{uniques}.

La \gls{query} suivante montre la création d'une table nommée PERSONNES, qui contient les attributs \textit{idpersonne}, \textit{nompersonne}, \textit{taillepersonne} et \textit{datenaissancepersonne}.

  \begin{lstlisting}
    CREATE TABLE PERSONNES
    (
    idpersonne CHAR(5),
    nompersonne VARCHAR(30),
    taillepersonne NUMBER,
    datenaissancepersonne DATE
    );
  \end{lstlisting}

\subsubsection{Alter table}
Permet de revenir sur l'instruction CREATE TABLE.

ALTER TABLE permet d'ajouter, supprimer ou modifier des \glspl{attribut}, des contraintes, des index, etc. appartenant à la table donnée en argument.
Cette instruction se comporte différemment selon qu'une table contiennent ou non des lignes de données (dits "\glspl{tuple}").
Par exemple, ajouter une contrainte NOT NULL sur une colonne possédant déjà des tuples nuls n'est pas possible.
Ce problème n'existe pas sur une table vide.

La \gls{query} suivante modifie la table \textit{PERSONNES} pour y ajouter une contrainte de clée primaire nommée \textit{pk\_personnes} sur l'attribut \textit{idpersonne}.

\begin{lstlisting}
  ALTER TABLE PERSONNES
  (
  ADD CONSTRAINT pk_personnes PRIMARY KEY (idpersonne)
  );
\end{lstlisting}

\subsubsection{Drop table}
Cette instruction permet de supprimer une table et les données qu'elle contient.
Dans une base de données relationnelles, les tables sont liées entre elles par des attributs.
La supression d'une table peut entraîner la supression de données d'autres tables, selon le schéma relationnel de la base.

La \gls{query} suivante supprime la table \textit{PERSONNES} de la base de données.
\begin{lstlisting}
  DROP TABLE PERSONNES;
\end{lstlisting}

\subsection{Le Langage de Manipulation des Données (LMD)}
Le langage de manipulation des données permet d'effectuer des actions de \gls{crud} sur ce que contiennent les tables.
En d'autres termes, il agit sur les \glspl{tuple}*.

\subsubsection{Create}
Il s'agit de créer un nouveau tuple dans une table.
La requête SQL suivante permet d'ajouter un tuple ayant pour clée primaire \textit{00001} dans la table \textit{PERSONNES}.
\begin{lstlisting}
  INSERT INTO PERSONNES
  (idpersonne, nompersonne, taillepersonne,
  datenaissancepersonne)
  VALUES
  ('00001', 'DUPONT', 'Jean', '06/08/1985');
\end{lstlisting}

\subsubsection{Read}
Il s'agit de récupérer et croiser les données que contiennent les tables.
Les requêtes SQL  "read" peuvent être très complexes.
Certains SGBD proposent des \glspl{qbe}
\footnote{Access, LOBase...}pour créer ces requêtes sans manipuler de \gls{sql}.
Celle qui est écrite juste après est simple et permet de retrouver le nom de la \textit{PERSONNES} numéro "00001".
\begin{lstlisting}
  SELECT PERSONNES.nompersonne
  FROM PERSONNES
  WHERE PERSONNES.idpersonne = '00001';
\end{lstlisting}

\subsubsection{Update}
Il s'agit de modifier un ou plusieurs tuples qui existent déjà dans la table.
La \gls{query} suivante remplace le nom de famille de la personne ayant pour clée primaire "00001" par "Robert".
\begin{lstlisting}
  UPDATE PERSONNES
  SET nompersonne = 'Robert'
  WHERE idpersonne = '00001';
\end{lstlisting}

\subsubsection{Delete}
Il s'agit de supprimer un ou plusieurs \glspl{tuple}.
La \gls{query} suivante supprime les tuples des \textit{PERSONNES} qui s'appellent "Jean".
\begin{lstlisting}
  DELETE FROM PERSONNES
  WHERE prenompersonne = "Jean";
\end{lstlisting}

\subsection{Le Langage de Contrôle des Données (LCD)}
Cet aspect du SQL n'est pas demandé pour l'application et n'est pas traité.

\subsection{Transaction ACID}
Cet aspect des bases de données n'est pas demandé pour l'application et n'est pas traité.
Par conséquent, une base de données géré par l'application ne peut être utilisée que par une seule personne à la fois, sinon elle perd sa cohérence.
Les modifications effectuées sur les données sont toutes automatiquement \gls{commit}* et l'utilisateur ne peut pas modifier ce comportement.

\subsection{Résumé des fonctionnalités}
L'application permet de :
\begin{itemize}
\item créer une table,
\item modifier une table,
\item supprimer une table,
\item ajouter des tuples,
\item croiser des données,
\item modifier des tuples,
\item supprimer des tuples.
\end{itemize}

Les fonctionalités précédentes sont utilisables sans connaissance du SQL.
