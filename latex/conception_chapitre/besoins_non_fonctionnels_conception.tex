Choix effectués pour répondre aux besoins non-fonctionnels.

\subsection{Systèmes d'exploitation}
L'application est développée avec le langage Java, pour profiter des \textit{machines virtuelles} qui permettent d'utiliser l'application sur n'importe quel système d'exploitation ayant une version du \gls{JRE}* installée
\footnote{\label{les_machines_virtuelles}Machines virtuelles Java : \url{https://docs.oracle.com/javase/8/docs/technotes/guides/vm/}}.

\subsection{Version du Java Runtime Environment}
L'\gls{IDE}* est configuré de façon à utiliser la version 1.7 de \gls{JRE}*.

\subsection{Interfaces Homme-Machine (IHM)}
Les IHM sont développées en Java, en utilisant les classes des packages \textit{java.swing} et \textit{javax.swing}. Ces packages disposent des listes déroulantes, cases à cocher, boutons, barre de défilements, etc.
\footnote{\label{element_de_formulaire}Package \textit{javax.swing} : \url{https://docs.oracle.com/javase/7/docs/api/javax/swing/package-summary.html}}.

\subsection{Performances}
Pour éviter les lenteurs dues aux accès vers le SGBD, l'application possède une couche \gls{orm} sur les tables.

\subsection{Compatibilité avec les SGBD}
L'application est conçue sur le principe \underline{ouvert-fermé} : pour la rendre compatible avec un nouveau SGBD, il n'y a que du code à \textit{ajouter}, aucun (ou presque) à \textit{modifier}.
