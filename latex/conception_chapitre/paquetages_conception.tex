Cette section détaille les couches évoquées dans l'autre section \ref{partie_architecture} et complète sur les paquetages \textit{factory} (visible sur la figure \ref{diagramme_de_paquetage_idb}) et \textit{useful}.

\subsection{Paquetage des fabriques}
\subsubsection{Utilité}
L'une des problématiques de ce projet est de faire fonctionner l'application par dessus tous les SGBD disponibles.
Bien que le langage SQL soit normé, les SGBD ne l'implémentent pas de la même manière, ce qui veut dire que la syntaxe des requêtes est \underline{différente} d'un SGBD à l'autre.
Ce point constitue la difficulté principale de l'application.
Il y en a d'autres mais elles sont mineures, comme la gestion de la casse
\footnote{\label{casse_et_sgbd}Oracle convertit le nom des tables et contraintes en majuscule, ce qui n'est pas le cas de MySQL par exemple.}.

En raison du temps disponible pour le développement, l'application ne fonctionne qu'avec les SGBD Oracle et MySQL.
En revanche, elle est conçue pour s'adapter à un nouvel SGBD sans modifier le code existant (ou presque) : il "suffit" d'en ajouter pour utiliser un nouvel SGBD.

Pour cela, l'application utilise une \textit{fabrique abstraite}, qui sur la figure \ref{diagramme_de_paquetage_idb}, se trouve dans le paquetage \textit{factory}.

\subsubsection{Dépendances supplémentaires}
Les dépendances de ce paquetage ne sont pas toutes représentées sur la figure.
Ce qui est visible dans le diagramme c'est que la fabrique est toujours appelée depuis un contrôleur (lui-même passant par les façades).
En plus de cela, la fabrique possède des dépendances de stéréotypes \textit{create} vers les paquetages :
\begin{itemize}
\item \textbf{gui} : \underline{pas} pour fabriquer des IHM, mais des classes imposées par Java pour les faire fonctionner, avec un comportement différent selon le SGBD connecté.
\item \textbf{manager} : pour créer des DAO générant le code SQL adapté au SGBD connecté.
\end{itemize}

\subsubsection{Statique de la fabrique}


\subsection{Paquetage des outils}
Plusieurs classes de cette application sont utilisées un peu partout dans le code.
Ces classes (développées par l'équipe du projet) sont regroupées dans le paquetage \textit{useful}, qui n'apparaît pas sur la figure \ref{diagramme_de_paquetage_idb}.

Tous les paquetages de cette application dépendent du paquetage useful, mais ce n'est pas génant dans la mesure où le code de ces classes n'est jamais amené à changer.
Ce paquetage peut être comparé à \textit{java.util}
\footnote{\label{paguetage_java_util}Package \textit{java.util} : \url{https://docs.oracle.com/javase/7/docs/api/java/util/package-summary.html}}
de java, qui contient les utilitaires que l'on retrouve dans toutes les applications.
