Les paquetages évoqués dans la section \ref{partie_architecture} sont détaillés dans cette section.
Le paquetage \textit{useful}, non visible sur la figure \ref{diagramme_de_paquetage_idb} est également détaillée.

\subsection{Paquetage des outils}
Plusieurs classes de cette application sont utilisées un peu partout dans le code.
Ces classes (développées par l'équipe du projet) sont regroupées dans le paquetage \textit{useful}, qui n'apparaît pas sur la figure \ref{diagramme_de_paquetage_idb}.

Tous les paquetages de cette application dépendent du paquetage useful, mais ce n'est pas génant dans la mesure où le code de ces classes n'est jamais amené à changer.
Ce paquetage peut être comparé à \textit{java.util}
\footnote{\label{paguetage_java_util}Package \textit{java.util} : \url{https://docs.oracle.com/javase/7/docs/api/java/util/package-summary.html}}
de java, qui contient les utilitaires que l'on retrouve dans toutes les applications.
