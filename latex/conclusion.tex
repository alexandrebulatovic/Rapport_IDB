Ce projet tuteuré demande le développement d’une application permettant d’utiliser une base de données sous n'importe quel SGBD sans utiliser le langage SQL.\\

L'application propose la gestion de base de données sous Oracle et MySQL. L'utilisateur peut créer, modifier, supprimer des tables dans sa base de données mais aussi lire, ajouter, modifier et supprimer les données de ces tables. Il peut aussi exécuter des requêtes graphiques simples (projections, sélections, jointures internes). Malgrès les différences entre chaque SGBD, l'application est construite de telle sorte que si un développeur souhaite implémenter un nouvel SGBD, il ne doit modifer aucune ligne du code actuel mais simplement en ajouter.\\

Le sujet du projet est lié aux cours que nous avons eus cette année. La \textit{Base de données} étant une des principales matières de la licence professionnelle ACPI, nous avons donc mis en application et approfondis les connaissances acquisent dans cette matière. De plus, nous avons appris de nombreux patterns de programmation dans le cours d'\textit{Architecture Logicielle}. En les appliquant au projet afin de faciliter l'implémentation des différents SGBD, nous avons acquis des facilités à les utiliser. En plus de ces connaissances, ce projet nous a permis d'approfondir nos connaissances en java et en la bibliothèque Swing. \\

Une bonne organisation et communication au sein d'un groupe de projet est essentiel au bon fonctionnement de celui-ci. En effet nous avons constaté, lors de certaines itérations, qu'un manque de communication n'était pas bénéfique au bon déroulement du projet. Cette constatation nous a permis de nous améliorer, au cours du projet, sur l'aspect organisationnel.