\newglossaryentry{table}
        {name={table},
        plural={tables},
        description={Ensemble d'\textit{attributs}.
        dans une base de données relationnelle, chaque ligne d'une table (nommée \textit{tuple}) est identifiée
        par un attribut ou groupe d'attributs
        uniques dans la table, que l'on nomme \textit{clée primaire}.
        Pour simplifier, on peut voir les tables comme des tableaux à une entrée : le nombre de colonnes est fixé, en revanche le nombre de lignes ne l'est pas}}

\newglossaryentry{tuple}
        {name={tuple},
        plural={tuples},
        description={Une ligne de données dans une table. Une table contient de 0 à n tuples}}
        
        
\newglossaryentry{JDBC}
{
  name=JDBC,
  description={Java DataBase Connectivity est une interface de programmation permettant de manipuler des bases de données avec des objets. Les systèmes de gestion de base de données doivent fournir un pilote JDBC correspondant à l'implémentation de cette interface}
}


\newglossaryentry{attribut}
        {name={attribut},
        plural={attributs},
        description={Un nom de colonne dans une table, également appelé \textit{champ}.
        Une table contient au moins un attribut}}
        
\newglossaryentry{data}{
        name={donnée},
        plural={données},
        description={Valeur brute, sans signification si aucun contexte n'est disponible. Du croisement des données résulte de l'information}}

\newglossaryentry{bdd}{
        name={base de données},
        plural={bases de données},
        description={Abrégé \textit{BD}. Outil permettant de stocker et de retrouver des ensembles des données.
        Dans ce rapport, ne sont mentionnées que des bases de données informatisées et relationelles}}

\newglossaryentry{bddr}{
        name={base de données relationnelle},
        plural={bases de données relationnelles},
        description={Abrégé \textit{BDR}. Il existe plusieurs types de bases de données.
        Dans ce rapport, ne sont mentionnées que des bases de données \textit{relationnelles}.
        Les données sont stockées dans des \textit{tables}, et les tables sont reliées entre elles par leurs clées primaires}}
        
        

\newacronym[plural={LDD},
        first={Langage de Définition des Données (LDD)},
        firstplural={Langages de Définition des Données}]
        {ldd}
        {LDD}
        {Langage de Définition des Données, un sous-ensemble du langage SQL permettant de décrire la structure des tables et les relations entre elles. Très grossièrement, le LDD permet de définir les tables et leurs colonnes}

\newacronym[plural={LMD},
        first={Langage de Manipulation des Données (LMD)},
        firstplural={Langages de Manipulation des données}]
        {lmd}
        {LMD}
        {Langage de Manipulation des Données, un sous-ensemble du SQL permettant d'effectuer les opérations CRUD sur les données contenues dans les tables}

\newacronym[plural={SQL},
        first={Structured Query Language (SQL)}]
        {sql}
        {SQL}
        {Structured Query Langage, un langage déclaratif et normé permettant d'utiliser les bases de données relationelles, largement inspiré par Codd en 1970 et devenu un standart aux Etats-Unis en 1984}

\newacronym[first={Create, Read, Update, Delete (CRUD)}]
                           {crud}
                           {CRUD}
                           {Create, Read, Update, Delete sont les quatre opérations basiques à effectuer sur des enregistrements de données, respectivement : en ajouter des nouveaux, les lire, les mettre à jour et les effacer.}

\newacronym[first={système d'Information (SI)},
                           firstplural={systèmes d'information (SI)}]
                           {si}
                           {SI}
                           {Le Système d'Information est un ensemble organisé de ressources qui permet de collecter, stocker, traiter et distribuer de l'information}

\newacronym[first={Système de Gestion de Base de Données (SGBD)},
                           firstplural={systèmes de gestion de base de données (SGBD)}]
                           {sgbd}
                           {SGBD}
                           {Système de Gestion de Base de Données, logiciel permettant la définition et la manipulation des bases de données. Dans ce rapport, ne sont mentionnés que des logiciel agissant sur des bases de données relationnelles}

\newglossaryentry{query}
        {name={requête SQL},
        plural={requêtes SQL},
        description={Une requête SQL est le nom couramment associé à du code SQL fonctionnel qui interroge ou agit sur la base de données}}

\newacronym[first={Interface Homme-Machine (IHM)},
	firstplural={interfaces homme-machine}]
	{ihm}
	{IHM}
	{Interface Homme-Machine, l'ensemble des moyens mis en oeuvre par l'homme pour communiquer avec la machine.
	Dans ce rapport, les IHM désignent les \textit{fenêtres} développées pour l'application}

\newglossaryentry{constraint}
        {name={contrainte},
        plural={contraintes},
        description={Restriction sur la saisie des données. Elles sont définies en SQL}}

\newglossaryentry{primarykey}
        {name={clée primaire},
        plural={clées primaires},
        description={Contrainte posée sur un attribut ou groupe d'attributs d'une table, les rendant identifiants, uniques et non nuls}}

\newacronym[first={Query By Example (QBE)},
        firstplural={Queries By Example (QBE)}]
        {qbe}
        {QBE}
        {Query By Example, IHM permettant de réaliser des requêtes SQL compliquées au moyens de clics de souris, de drag-and-drop et autres
        facilités ne demandant pas de connaître le langage SQL}
		
\newglossaryentry{IDE}
		{name={environnement de développement intégré},
		plural={environnements de développements intégrés},
		description={ensemble d'outils nécessaire à la programmation, regroupés en un logiciel (compilateur, éditeur de liens, débogeur, auto-complétion etc)}}

\newacronym[first={Object Relational Mapping (ORM)}]
        {orm}
        {ORM}
        {Object Relational Mapping, technique visant à convertir des groupes de données vers des instances, pour les manipuler avec un langage objet}

\newglossaryentry{mock}
		{
		name={mock},
		plural={mocks},
		description={objets simulés qui reproduisent le comportement d'objets réels de manière contrôlée. Utile pour tester individuellement des classes ayant besoin d'un classe tièrce suseptible de bugger à son tour.
		Le test doit se porter sur la classe sans éléments perturbateurs}
		}

\newacronym[first={Développement Piloté par les Tests (TDD)},
	firstplural={Développements Pilotés par les Tests}]
	{tdd}
	{TDD}
	{Le Développement Piloté par les Tests(\textit{test-driven development}), 
	est une technique de développement de logiciel qui préconise d'écrire les \textbf{tests unitaires} avant d'écrire le \textbf{code source} d'un logiciel.}

\newacronym[first={Data Access Object (DAO)},
                        firstplural={Data Access Objects}]
        {dao}
        {DAO}
        {Data Access Object, couche de l'application permettant d'enregistrer les données sur un système de stockage, comme par exemple une base de données}
