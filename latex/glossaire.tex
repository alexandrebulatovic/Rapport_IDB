\newglossaryentry{data}{
        name={donnée},
        plural={données},
        description={Valeur brute, sans signification si aucun contexte n'est disponible. Du croisement des données résulte de l'information}}

\newglossaryentry{bdd}{
        name={base de données},
        plural={bases de données},
        description={Abrégé \textit{BD}. Outil permettant de stocker et de retrouver des ensembles de \glspl{data}.
        Dans ce rapport, ne sont mentionnées que des bases de données informatisées}}

\newglossaryentry{bddr}{
        name={base de données relationnelle},
        plural={bases de données relationelles},
        description={Abrégé \textit{BDR}. Il existe plusieurs types de \glspl{bdd}.
        Dans ce rapport, ne sont mentionnées que des bases de données \textit{relationnelles}.
        Les données sont stockées dans des \textit{tables}, c'est à dire des ensembles d'\textit{attributs} (ou \textit{colonnes}).
        Chaque ligne d'une table (nommé \textit{tuple}) est identifiée par un attribut ou groupe d'attributs \underline{uniques} dans la table.
        Les tables sont reliées entre elles par ces groupes d'attributs uniques.
        Pour simplifier, on peut voir les tables comme des tableaux à une entrée : le nombre de colonnes est fixé, en revanche le nombre de lignes ne l'est pas}}

\newacronym[plural={LDD},
        first={Langage de Définition des Données},
        firstplural={Langages de Définition des Données}]
        {ldd}
        {LDD}
        {Langage de Définition des Données, un sous-ensemble du langage \gls{sql} permettant de décrire la structure des tables et les relations entre elles. Très grossièrement, le LDD permet de définir les tables et leurs colonnes}

\newacronym[plural={LMD},
        first={Langage de définition des données (LMD)},
        firstplural={Langages de définiton des données}]
        {lmd}
        {LMD}
        {Langage de Manipulation des Données, un sous-ensemble du \gls{sql} permettant d'effectuer les opérations \gls{crud} sur les données contenues dans les tables}

\newacronym[plural={SQL},
        first={Structured Query Language (SQL)}]
        {sql}
        {SQL}
        {Structured Query Langage, un langage déclaratif et normé permettant d'utiliser les \gls{bddr}, largement inspiré par Codd en 1970 et devenu un standart aux Etats-Unis en 1984}

\newacronym[first={Create, Read, Update, Delete (CRUD)}]
                           {crud}
                           {CRUD}
                           {Create, Read, Update, Delete sont les quatre opérations basiques à effectuer sur des enregistrements de données, respectivement : en ajouter des nouveaux, les lire, les mettre à jour et les effacer.}

\newacronym[first={système d'information (SI)},
                           firstplural={systèmes d'information (SI)}]
                           {si}
                           {SI}
                           {Le Système d'Information est un ensemble organisé de ressources qui permet de collecter, stocker, traiter et distribuer de l'information}

\newacronym[first={système de gestion de base de données (SGBD)},
                           firstplural={systèmes de gestion de base de données (SGBD)}]
                           {sgbd}
                           {SGBD}
                           {Logiciel permettant la définition et la manipulation des bases de données. Dans ce rapport, ne sont mentionnés que des logiciel agissant sur des bases de données relationnelles}
