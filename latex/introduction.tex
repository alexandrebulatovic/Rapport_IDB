Les \gls{bdd} sont des outils permettant de stocker et retrouver des \glspl{data}, c'est à dire des valeurs brutes. Ces outils se trouvent au coeur des \glspl{si}, et sont donc indispensables aux entreprises.

En informatique, les bases de données sont informatisée, définies et manipulées grâce à des logiciels nommés \glspl{sgbd}.

Il existe différents types de bases de données, mais le marché reste dominé par les \glspl{bddr} \footnote{\label{part_de_marché_relationnel}Classement des \glspl{sgbd} les plus populaires : \url{http://db-engines.com/en/ranking}}. Ces dernières sont gérées par des SGBD relationels. Parmi les plus connus, on trouve :
\begin{itemize}
\item Oracle,
\item MySQL,
\item PostgreSQL.\\
\end{itemize}

Ces logiciels proposent tous de définir et de manipuler des bases de données par le biais du \gls{sql}, un langage déclaratif et normé, implémenté par les SGBD avec des écarts vis à vis de ce que préconise la norme\footnote{\label{differences_implementation_sql_sqgbd}Comparer les documentations Oracle et MySQL pour modifier une table \url{https://docs.oracle.com/cd/B28359_01/server.111/b28286/statements_3001.htm} contre \url{https://dev.mysql.com/doc/refman/5.7/en/alter-table.html}}. De ce fait, deux SGBD distincts ne proposent pas \textit{exactement} le même SQL.

Se posent alors deux problématiques : 
