Les tests constituent des éléments indispensables, ils permettent d'avancer sur un projet sans que les nouvelles implémentations ne viennent modifier le comportement existant.
\section{Pourquoi tester}


Idéalement, chaque classe devrait être testée \textbf{unitairement}.
Malheureusement, certains comportement ne peuvent pas être testés  de façon automatique. Par exemple, pour tester les fonctionnalités d'un logiciel ayant des interactions avec \gls{sgbd}, il faudrait que l'utilisateur stocke ses identifiants de connexion en dur dans une classe accessible par sa classe de test, et il faudrait également qu'il ait préalablement créé manuellement la liste de toutes les tables nécessaires au fonctionnement de son programme.
\bigbreak

Pour pallier à ce problème, il est possible de créer une classe mock avec des comportements précis directement introduits par le développeur. Si la classe normale doit effectuer une vérification sur un système externe et retourner \textit{true} ou \textit{false} selon le résultat de cette vérification, la classe mock, elle, retournera directement \textit{true} ou \textit{false} sans faire de vérifications, c'est ce que l'on appelle "\underline{mocker la classe}"). L'inconvénient de cette façon de faire est que cela agrandit considérablement le nombre de classes Java du projet.


\bigbreak
  Heureusement, il existe des bibliothèques externes qui permettent de créer des objets \gls{mock}* sans avoir à créer de nouvelles classes.
Les classes natives de Java ne seront pas "mockés" car celles-ci ont déjà été testées avant d'être intégrées à Java et sont donc considérées "sans bugs".
\bigbreak
Exemple d'utilisation de la bibliothèque externe \textit{Mockito} :

\lstset{
language=Java,
basicstyle=\normalsize, % ou ça==> basicstyle=\scriptsize,
upquote=true,
aboveskip={1.5\baselineskip},
columns=fullflexible,
showstringspaces=false,
extendedchars=true,
breaklines=true,
showtabs=false,
showspaces=false,
showstringspaces=false,
identifierstyle=\ttfamily,
keywordstyle=\color[rgb]{0,0,1},
commentstyle=\color[rgb]{0.133,0.545,0.133},
stringstyle=\color[rgb]{0.627,0.126,0.941},
}
\begin{lstlisting}
SQLManager sqlManager = Mockito.mock(SQLManager.class);

Mockito.when(sqlManager.sendQuery(Mockito.contains("SELECT"))).thenReturn(true);
Mockito.when(sqlManager.sendQuery("")).thenThrow(IllegalArgumentException.class);
Mockito.when(sqlManager.getGeneratedJTable()).thenReturn(new JTable());
Mockito.when(sqlManager.getGeneratedReply()).thenReturn(Mockito.anyString());

\end{lstlisting}
\bigbreak

Il est utile de tester pour vérifier que les fonctionnalités telles que définies dans le cahier des charges correspondent bien à celles qui ont été développées.
\exemple{Un utilisateur doit pouvoir se connecter à la base de données. Si un jour, après une légère modification, l'utilisateur ne peut plus se connecter alors qu'il le devrait, le reste du logiciel deviendrait inutile aux yeux du client\newline}

Certaines classes ne sont pas testables sans utiliser des bibliothèques complexes nécessitant d'écrire de nombreuses de lignes de code pour tester une seule fonctionnalité. C'est le cas notamment des classes servant d'IHM et dont la seule façon possible de tester est d'implémenter un objet dit "Robot" qui simulerait l'action d'un utilisateur humain. Après concertation avec le tuteur du projet il a été décidé que ces tests se feront manuellement.

\section{Les tests choisis}

Nous avons donc choisis de tester en priorité les classes \textbf{Métiers}, une suite de tests appelés \textbf{AllTests} se charge  de lancer tous les tests liés aux classes métiers :
\\

\begin{itemize}
	\item testAttribute
	\item testTable
	\item testConstraint
	\item testTableSet
\end{itemize}

\textit{Voir section \ref{section_metiers} traitant des classes métiers}
\\


La plupart des tests ont été écrits après l'implémentation des fonctionnalités sauf certaines classes utilisant les classes du package \textbf{Métiers} qui ont été développés en \gls{tdd}*.
\medbreak
La classe faisant office de \gls{dao}* pour le \gls{crud} utilise une bibliothèque externe pour pouvoir mocker son comportement : \textbf{Mockito}. Tous les appels à cette classe sont donc capturés et gérés par le développeur directement dans la classe de test une fois celle-ci "mocké", ce qui permet de tester sa couche de contrôle et sa couche de façade sans créer de classes de \gls{mock} supplémentaires.

\medbreak

Les structures de données ResponseData et Response servant à éviter de traiter des exceptions \textit{(voir Figure  \ref{uml_classe_response})} sont également testés individuellement, leur place étant importante au sein du projet.

\bigbreak
Des tests sur la connexion sont également effectués, c'est l'un des éléments clés du projet car toutes les fonctionnalités reposent sur le fait que la connexion soit fonctionnelle. Les tests d'intégrations ne passent pas si les tests unitaires de la connexion ne passent pas).

Pour cela, différents objets nécessaires à l'établissement d'une connexion ont été \textit{mockés} et des tests vérifiant l'état de la connexion sont effectués.

