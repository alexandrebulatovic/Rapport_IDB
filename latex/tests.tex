Les tests constituent des éléments indispensables permettant la progression dans un projet en toute sécurité.
\section{Pourquoi Tester}

Idéalement, chaque classe (donc chaque fichier en Java), devrait être testé \textbf{individuellement}. C'est-à-dire qu'il faut créer des classes \Glspl{mock} pour isoler chaque classe testé.
Les classes natives de Java ne seront pas "Mocké" car elles sont considéré "parfaites".

Il est utile de tester pour vérifier la validations de point de fonctionnalités définies sur un cahier des charges.
\exemple{L'utilisateur doit pouvoir se connecter à la base de données, si un jour, après une légère modification, l'utilisateur ne peut plus de connecter, le reste du logiciel deviendra inutile aux yeux du client\newline}

Bien entendu, il n'est pas forcément utile de tester toutes les classes, mais seulement les plus importantes et pertinentes.

\section{Les Tests Choisis}

Nous avons donc choisi de tester en priorité les classes \textbf{Métiers} : Une classe de Test noté \textbf{AllTests.class} se charge alors de lancer tous les tests liés aux classes métiers : 
\begin{itemize}
	\item testAttribute
	\item testTable
	\item testConstraint
	\item testTableSet
\end{itemize}

\textit{Voir section \ref{section_metiers} traitant des classes métiers}


Nous avons mélangé les tests en \textbf{Pro Ingénierie} et les tests en \textbf{Rétro Ingénierie}. 
Lorsque les classes externes ont eu besoin des classes métiers, certaines fonctionnalités ont alors été ajouté en \gls{tdd}.

Pour finir, les tests se sont portés sur la connexion, l'élément le plus important dans notre projet, 
car elle est utilisé indirectement par toutes les autres classes (aucun test d'intégration ne passerait 
si les tests unitaires de la connection ne passaient pas).

Pour cela, un \Gls{mock} à été créé afin de se passer du DDLContrôleur.

