\section{Télécharger GIT}
GIT est le gestionnaire de version du projet. Il est disponible pour windows comme pour ubuntu.
\subsection{Télécharger sous ubuntu}
\begin{itemize}
\item Ouvrir un terminal \textbf{ctrl + alt + t};
\item Taper la commande \textbf{sudo apt-get install git}.
\end{itemize}

\subsection{Télécharger sous windows}
Se rendre à l'adresse \url{https://git-scm.com/download/win}. Le téléchargement devrait se lancer seul. Si ce n'est pas le cas, choisir la version de GIT voulue.

Installer GIT comme une application courante, en lisant attentivement les paramètres de compatibilités avec Linux.

Faire en sorte d'avoir à disposition un terminal linux (\textit{Bash}), et éventuellement celui de windows (\textit{CMD}).

\section{Créer sa zone de travail}
\subsection{Ouvrir le terminal}
\begin{itemize}
\item ubuntu : \textbf{ctrl + alt +t}.
\item windows : \textbf{Démarrer, "git bash"}.
\end{itemize}

\subsection{Se déplacer vers le répertoire voulu}
Utiliser la commande \textbf{cd <chemin\_acces\_repertoire>} pour déplacer le répertoire courant du terminal à la destination voulue. Le chemin d'accès \textbf{..} permet de revenir au dossier parent.

La commande \textbf{ls -la} affiche l'intégralité du répertoire courant.

La commande \textbf{pwd} affiche le chemin d'accès absolu du répertoire courant.

\subsection{Créer le workspace Eclipse}
Une fois sur le bon répertoire :
\begin{enumerate}
\item \textbf{mkdir tuteur}, ce qui fabrique un nouveau répertoire \textit{tuteur};
\item ouvrir Eclipse;
\item positionner le workspace Eclipse dans le répertoire \textit{tuteur};
\item créer un \underline{nouveau projet java} nommé \textit{idb}.
\end{enumerate}

\section{Récupérer le projet}
Revenir sur le terminal.
\begin{enumerate}
\item \textbf{cd tuteur/idb};
\item \textbf{rm -fR src};
\item \textbf{git clone https://github.com/alexandrebulatovic/IDB};
\item \textbf{mv IDB src}
\end{enumerate}

Revenir sur Eclipse, cliquer droit sur l'arborescence du projet, cliquer sur \textit{Refresh}.

C'est terminé.



