\subsection{Créer une table sans contrainte}

\begin{description}
\item[Acteur] : l'utilisateur du logiciel (un étudiant ou un professeur).
\item[Objectif] : créer une ou plusieurs tables dans la base de données.
\item[Pré-conditions] : l'utilisateur s'est connecté à la base de données cible.
\item[Post-conditions] : la ou les tables sont enregistrées dans la base de données ou sinon un message incite à corriger l'erreur.
\end{description}

\textbf{--> Scénario nominal :}
\begin{enumerate}
\item Le système affiche une IHM particulière.
\item L'utilisateur renseigne le nom de la table, les noms des attributs et leurs types.
\item L'utilisateur envoie sa table au système.
\item Le système retourne un message de réussite.
\end{enumerate}

\textbf{--> Scénario d'exceptions :}
\begin{description}
\item 4)a) La table n'est pas créée : les attributs contiennent des caractères spéciaux.
\item 4)b) La table existe déjà.
\end{description}

\subsection{Créer une table avec contraintes}
\begin{description}
\item[Acteur] : l'utilisateur du logiciel (un étudiant ou un professeur)
\item[Objectif] : créer une ou plusieurs tables dans la base de données avec éventuellement des contraintes
\item[Pré-conditions] : l'utilisateur s'est connecté à la base de données cible
\item[Post-conditions] : la ou les tables sont enregistrées dans la base de données ou sinon un message incite à corriger l'erreur.
\end{description}

\textbf{--> Scénario nominal :}
\begin{enumerate}
\item Le système affiche une IHM particulière.
\item L'utilisateur renseigne le nom de la table, les noms des attributs et leurs types.
\item L'utilisateur choisit les contraintes (NOT NULL, UNIQUE, FOREIGN KEY, PRIMARY KEY, CHECK.) associées aux attributs.
\item L'utilisateur envoie sa table au système.
\item Le système retourne un message de réussite.
\end{enumerate}

\textbf{--> Scénario d'exceptions :}
\begin{description}
\item 4)a) Le nom des attributs contient des caractères spéciaux.
\item 4)b) La requête n'a pas réussi car la table existe déjà.
\item 4)c) La clé étrangère n'existe pas, le système demande à ce que la table référencée soit créée avant d'indiquer la clé étrangère.
\end{description}
