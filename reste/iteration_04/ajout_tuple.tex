\documentclass[a4paper, 12pt]{article}

\usepackage[french]{babel}
\usepackage[T1]{fontenc}
\usepackage[utf8]{inputenc}
\usepackage{lmodern}
\usepackage{graphicx}
\usepackage{hyperref}
\usepackage{listings}
\usepackage{graphicx}
\usepackage{microtype}
\begin{document}
\section{Ajouter un tuple à une table}
\begin{description}
\item[Acteur] : un utilisateur lambda.
\item[Objectif] : ajouter un tuple à une table.
\item[Pré-conditions] : l'utilisateur est connecté avec succès à son SGBD depuis l'application.
\end{description}

\subsection{Scénario nominal}
\begin{description}
\item[1-] L'IHM propose la liste des tables présente sur la base de données de l'utilisateur.
\item[2-] L'utilisateur sélectionne une table.
\item[3-] Les attributs de la table apparaissent avec des champs de texte.
\item[4-] L'utilisateur écrit les données qu'il veut insérer.
\item[5-] L'utilisateur valide ses données.
\end{description}

\subsection{Scénario alternatif}
\begin{description}
\item[5-a] L'utilisateur ferme la fenêtre pour annuler l'insertion.
\end{description}

\subsection{Scénario d'erreur}
\begin{description}
\item[1-b] La récupération des tables à échouée, un message indique l'erreur.
\item[5-a] L'utilisateur n'a pas respecté les contraintes de sa table, un message indique l'erreur.
\end{description}
\end{document}
